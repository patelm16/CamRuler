\documentclass{article}

\usepackage{booktabs}
\usepackage{tabularx}
\usepackage{hyperref}

\title{SE 3XA3: Development Plan\\CamRuler}

\author{Team 10, CamRuler
		\\ Prince Kowser kowserm
		\\ Meet Patel patelm16
		\\ Kshitij Mehta mehtak1
}

\date{}



\begin{document}

\begin{table}[hp]
\caption{Revision History} \label{TblRevisionHistory}
\begin{tabularx}{\textwidth}{llX}
\toprule
\textbf{Date} & \textbf{Developer(s)} & \textbf{Change}\\
\midrule
09/28/2017 & Prince, Meet, Kshitij & Rev 0\\
\bottomrule
\end{tabularx}
\end{table}

\newpage

\maketitle

The app we are re-implementing is a camera ruler, which allows the user to measure an object in front of them through a picture of the object and a reference object. Having this app on mobile devices will enable users to conveniently measure objects without the struggle of finding a ruler and doing tedious measurements by hand. 

\section{Team Meeting Plan}
-When : Tuesdays and Wednesdays around 6.30pm
\\-Where : Mills 4th floor
\\-Frequency : twice in one week
\\-Roles : Scriber, Leader, Researcher 
\\-Rules for agenda : 
\begin{itemize}
 \item Review last meeting 
 \item Ask for any new input or changes
 \item Brainstorm proccess for next deliverable
 \item Finalize dates and roles
 \item Progression check
\end{itemize}

  
\section{Team Communication Plan}
 \begin{itemize}
 \item Git for uploading code and documents
 \item Facebook group chat for communication outside university and planning 
 \item Google docs for working on documents together 

\end{itemize}

\section{Team Member Roles}
During each meeting, we will have an assigned team leader to make sure all the work is present and on git, figure out meeting time and dates, and divide work. We will also have a writer to record what is said during the meeting into a meeting minutes document. The other individual will be researching and preparing for next meeting. In our group, we have individuals who have more expertise in git, Latex, and Andriod Development. If we run into issues, these individuals will be asked to solve the problem prior to asking the TA or Professor. 

\section{Git Workflow Plan}
We will be using a centralized workflow and from their we have developer brancches for each developer to work on their own code and then also a feature branch that allows each developers to work on any cool feature from any of the developers branches. We will use git milestones to set goals and deadlines and when they are reached, we use a git label to point at that code in time. Git labels will also be used in order to tag significant changes.  

\section{Proof of Concept Demonstration Plan}
We had trouble choosing our project idea because we saw many risks associated with our previous ideas. With this project, the testing and the use of libraries should not be difficult as we already have the code. However, our main risk is that two of us have no experience using Android Studio while one of us knows a little bit about it. All in all, we are very new to using Android Studio which may cause a few issues. However, we plan on doing a lot of research online to play around with Android Studio. If we see that after one month, we still feel uncomfortable with Android Studio, we will reduce our scope and get more help from the TAs. 

\section{Technology}
We will be coding with Java using Android Studio to create our Android application, while using Eclispe as our IDE. To generate our documents, we will be using Latex. We will be using JUnit to test our code and to test the actual application, we will be using an Android device. We will also be using integration testing with Android Studio.

\section{Coding Style}
We will adhere to the standard Java coding style for our project, which will follow the correct conventions for naming, file structure, formatting, and correct programming practices. 

\section{Project Schedule}

Please see our Gantt chart \href{https://gitlab.cas.mcmaster.ca/kowserm/3xa3/blob/master/Doc/Design/Gantt%20Chart/project_gantt.gan}{here}

\section{Project Review}

\end{document}