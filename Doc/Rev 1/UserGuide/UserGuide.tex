\documentclass{article}
\usepackage{tabularx}
\usepackage{booktabs}
\usepackage[normalem]{ulem}
\usepackage{hyperref}

\begin{document}

\title{SE 3XA3: User Guide\\CamRuler}

\author{Team 10, CamRuler
		\\ Kshitij Mehta, mehtak1
		\\ Prince Kowser, kowserm
		\\ Meet Patel, patelm16
}

\date{\today}

\newpage

\maketitle

\section{Installing the App and Setting Up Android Studio On the Computer}
The Android application we have developed is called CamRuler and is made using Android Studio and Java. To test this app on your computer, you must have Android Studio downloaded on your computer. Then you want to import the project into Android Studio and run it on the emulator. If you do not have Android Studio downloaded or need help with the setup, check out this \href{https://developer.android.com/training/index.html}{link.}

\section{Installing the App on Your Android Mobile Device}
To install CamRuler on your Android mobile device, make sure that it is v4.0 or higher. Next you want to put your phone in developer mode and check the option to enable USB Debugging. Make sure the project is open on Android Studio before pressing play to side load the app onto your Android mobile device. 

\section{How to Use the App}
This application can be used to calculate the measurements of an object. To use this application, follow these steps:

1. After clicking the Start button, chose the number of objects you want to measure through the picture you are going to take.

2. Take a picture of the object(s) to be measured and along with the reference object.

3. Select both the objects by drawing dots on the two ends. Yellow line represents reference object and red line represents the actual object. The blue line is for the points of the second object.

4. Enter in the measurements of the reference object along with the units of the reference object and the desired units to be outputted.

5. Click the OK button and the measurements for the measurements.

**You always have the option to go back in case you have made an error such as taking the bad picture or selecting the wrong points.

\end{document}