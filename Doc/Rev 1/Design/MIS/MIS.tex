\documentclass[12,english]{article}
\usepackage[letterpaper, portrait, margin=1in]{geometry}

\usepackage{amsmath}
\usepackage[T1]{fontenc}
\usepackage{babel}
\usepackage{textcomp}
\usepackage{titlesec}
\setcounter{secnumdepth}{4}
\usepackage{hyperref}
\usepackage{xcolor}
\usepackage{booktabs}
\usepackage{placeins}
\usepackage{multirow}


\hypersetup{
    bookmarks=true,         % show bookmarks bar?
      colorlinks=true,       % false: boxed links; true: colored links
    linkcolor=black,          % color of internal links (change box color with linkbordercolor)
    citecolor=green,        % color of links to bibliography
    filecolor=magenta,      % color of file links
    urlcolor=cyan           % color of external links
}

\titleformat{\paragraph}
{\normalfont\normalsize\bfseries}{\theparagraph}{1em}{}
\titlespacing*{\paragraph}
{0pt}{3.25ex plus 1ex minus .2ex}{1.5ex plus .2ex}

\title{Module Interface Specfication for Group 10 - CamRuler}
\author{Prince Kowser\\
\texttt{ (kowserm)}
\and
Meet Patel\\
\texttt{(patelm)}
\and
Kshitij Mehta\\
\texttt{(mehtak1)}
}

\date{}


\begin{document}
\maketitle
\newpage
\tableofcontents
\newpage

\section{Module Hierarchy}
\begin{table}[!htbp]
        \begin{tabular}{ll}
        \toprule
        Level 1 & Level 2 \\
        \midrule
        Hardware Hiding Module & \\
         \midrule
        Behaviour Hiding Module & DrawView Module \\
        & ImageSurface Module\\
        & InputDialog Module\\
        & Ruler Module\\ 
		& Utils Module \\
		 \midrule
        Software Decision Module & Main Activity Module\\
        
        \bottomrule
        \end{tabular}
        \caption{Module Hierarchy}
        % Colour for the rulings in tables:
        \makeatletter
           \def\rulecolor#1#{\CT@arc{#1}}
           \def\CT@arc#1#2{%
           \ifdim\baselineskip=\z@\noalign\fi
           {\gdef\CT@arc@{\color#1{#2}}}}
           \let\CT@arc@\relax
          \rulecolor{black!50}
        \makeatother
        \label{Table 1}
        \end{table}

\section{MIS of DrawView Module}
		\subsection{Interface Syntax}
		\subsubsection{Exported Access Programs}
		\begin{tabular}[pos]{|c|c|c|c|}
			
			\hline
			%	\label
			\textbf{Name}& \textbf{In} & \textbf{Out} & \textbf{Exceptions} \\ \hline
			DrawView & context & -  & -\\ \hline
			onDraw & canvas & -  & -\\ \hline
			onTouchEvent & event & -  & -\\ \hline
			clearCanvas & clearCanvas & -  & -\\ \hline
			Calculate & double, int, int & double  & -\\ \hline
		\end{tabular}
		
		\subsection{Interface Semantics}
		\subsubsection{State Variables}
		paint: paint - holds drawin line information
		circlePoints: List<Point> - stores user's drawn points 
		Context: context - Holds the context of current state of the object
		Referencepointcolor: Color - holds color information
		Measurepointcolor: Color - holds color information
		
		\subsubsection{Environmental Variables}
		Not Applicable
		
		\subsubsection{Assumptions}
		Variables should be set before trying to access them
		
		\subsubsection{Access Program Semantics}
		DrawView(context):
		
		Input: context
		
		Transition: Prepares the context of the object to be drawn on
		
		Output: None\\
		Exceptions: None\\
		\\
		onDraw(Canvas):
		
		Input: Canvas the object to draw on
		
		Transition: Draws the line between two points after user inputs the reference points
		
		Output: none
		
		Exception: none\\ 
		\\
		onTouchEvent(event):
		
		Input: event
		
		Transition: Restricts the user to draw only 4 points and manages the transition between each point drawn
		
		Output: none
		
		Exception: none\\ 
		\\
		clearCanvas():
		
		Input: 
		
		Transition: Clears the canvas of all points drawn
		
		Output: none
		
		Exception: none\\ 
		\\
		Calculate():
		
		Input: double length of reference object, int input unit index, int output unit index 
		
		Transition: Calculates the measurement
		
		Output: the measuremnt of the drawn object
		
		Exception: none\\ 
		\\
	
	
	
\section{MIS of ImageView}
		\subsection{Interface Syntax}
		\subsubsection{Exported Access Programs}
		\begin{tabular}[pos]{|c|c|c|c|}
			
			\hline
			%	\label
			\textbf{Name}& \textbf{In} & \textbf{Out} & \textbf{Exceptions} \\ \hline
			ImageSurface & context, image & - & -\\ \hline
			onDraw & canvas & - & -\\ \hline
			Surfacecreated & holder &  & -\\ \hline

			
		\end{tabular}
		
		\subsection{Interface Semantics}
		\subsubsection{State Variables}
	    icon: Bitmap - used to store point drawing info
		paint: Paint - ink info\\

		\subsubsection{Environmental Variables}
		
		\subsubsection{Assumptions}
		Variables should be set before trying to access them
		
		\subsubsection{Access Program Semantics}
		
		ImageSurface(context, image):
		
		Input: context, image
		
		Transition: Craetes a surface to draw on the image
		
		Output: none
		Exceptions: None\\
		\\
		onDraw(canvas):
		
		Input: canvas - the object to draw on
		
		Transition: draws the two points 
		
		Output: none
		
		Exception: none\\ 
		\\
		surfacecreated(holder):
		
		Input: holder - holds the surface information 
		
		Transition: unlocks the drawing surface when user input is needed or else keep it locked
		
		Output: none
		
		Exception: none\\ 
		
	
	
	
\section{MIS of InputDialog}
	\subsection{Interface Syntax}
		\subsubsection{Exported Access Programs}
		\begin{table}[!htbp]
		\begin{tabular}{|c|c|c|c|}
			\hline
			onCreateDialog & Bundle & Dialog & - \\ \hline
		\end{tabular}
	\end{table}
		
	\subsection{Interface Semantics}
		\subsubsection{State Variables}
		
		\subsubsection{Environmental Variables}
		
		\subsubsection{Assumptions}
			
		\subsubsection{Access Program Semantics}

onCreateDialog(savedInstanceState):

		Input: no inputs
		
		Transition: Creates a pop up dialog on screen
		
		Output: the dialog
		
		Exception - None\\
		



\section{MIS of Calculation}
	\subsection{Interface Syntax}
		\subsubsection{Exported Access Programs}
		
	\begin{tabular}[pos]{|c|c|c|c|}
	\hline
	\textbf{Name}& \textbf{In} & \textbf{Out} & \textbf{Exceptions} \\ 
	\hline
	compute & List<point>, double, Int, Int & double & -\\ 
	\hline
	getDistance & Point,Point & double & -\\ 
	\hline
	convertUnits & int,double,int,double & double & -\\ 
	\hline
	toMeters & double,int & double & -\\ 
	\hline
		
	\end{tabular}		
		
	\subsection{Interface Semantics}
		\subsubsection{State Variables}
		Not Applicable
		\subsubsection{Environmental Variables}
	
		\subsubsection{Assumptions}

		\subsubsection{Access Program Semantics}
		compute(points, scale, inputindex, outputindex):
		
		Input: List<Point> points, double scale, int inputUnitIndex, int outputUnitIndex
		
		Transition: Calculates the actual measurement in according to reference object 
		
		Output: the measurement 
		
		Exception: \\
		\\
		getDistance(point,point):
		
		Input: two Point objects, p1 and p2
		
		Transition: calculate distance between each point
		
		Output: the distance
		
		Exception:
		\\
		conveertunits(refunit,reference,meaunit,measurement):

		Input: reference input numbere, reference measurement, measuremnt output unit, masurement of oject
		
		Transition: do the conversion math
		
		Out: the converted measurement
		
		Exception: 
		\\
	
		toMeters(measurement, refunit):
		
		Input: measurement and reference unit indicater 
		
		Transition: convert to meters
		
		Output: converted measurment
		
		Exception: none\\
		\\
		
\section{Major Revision History}

November 10, 2017 - Revision 0 roughdraft

			
\end{document}
