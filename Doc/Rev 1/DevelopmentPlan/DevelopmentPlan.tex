\documentclass{article}

\usepackage{booktabs}
\usepackage{tabularx}

\usepackage[normalem]{ulem}
\usepackage[colorlinks]{hyperref}


\title{SE 3XA3: Development Plan\\CamRuler}

\author{Team 10, CamRuler
		\\ Prince Kowser kowserm
		\\ Meet Patel patelm16
		\\ Kshitij Mehta mehtak1
}

\date{}



\begin{document}

\begin{table}[hp]
\caption{Revision History} \label{TblRevisionHistory}
\begin{tabularx}{\textwidth}{llX}
\toprule
\textbf{Date} & \textbf{Developer(s)} & \textbf{Change}\\
\midrule
September 28, 2017 & Prince, Meet, Kshitij & Rev 0\\
\midrule
December 6, 2017 & Kshitij Mehta, Meet Patel, Prince Kowser & Rev 1\\
\bottomrule
\end{tabularx}
\end{table}

\newpage

\maketitle

The app we are re-implementing is a camera ruler, which allows the user to measure an object in front of them through a picture of the object and a reference object. Having this app on mobile devices will enable users to conveniently measure objects without the struggle of finding a ruler and doing tedious measurements by hand. 

\section{Team Meeting Plan}
\begin{itemize}
\item When : Tuesdays and Wednesdays around 6.30pm
\item Where : Mills 4th floor
\item Frequency : twice in one week
\item Roles : Scriber, Leader, Researcher 
\item Rules for agenda : 

\begin{itemize}
 \item Review last meeting 
 \item Ask for any new input or changes
 \item Brainstorm proccess for next deliverable
 \item Finalize dates and roles
 \item Progression check
\end{itemize}

\end{itemize}

  
\section{Team Communication Plan}
 \begin{itemize}
 \item {\color{red}GitLab} for uploading code and documents
 \item Facebook group chat for communication outside university and planning 
 \item \sout{Google docs} {\color{red}GitLab} for working on documents together 
\end{itemize}

\section{Team Member Roles}
During each meeting, we will have an assigned team leader to make sure all the work is present and on {\color{red} GitLab}, figure out meeting time and dates, and divide work. We will also have a {\color{red} scriber} to record what is said during the meeting into a meeting minutes document. The other individual will be researching and preparing for next meeting. In our group, we have individuals who have more expertise in {\color{red} Git} {\color{red} (Kshitij)}, LaTeX {\color{red}
(Meet)}, and Andriod Development {\color{red} (Prince)}.If we run into issues, these individuals will be asked to solve the problem prior to asking the TA or Professor. 

\section{Git Workflow Plan}
We will be using a centralized workflow and from their we have developer brancches for each developer to work on their own code and then also a feature branch that allows each developers to work on any cool feature from any of the developers branches. We will use {\color{red} Git} milestones to set goals and deadlines and when they are reached, we use a {\color{red} Git} {\color{red} tag} to point at that code in time. Git {\color{red} tags} will also be used in order to indicate significant changes.  

\section{Proof of Concept Demonstration Plan}
We had trouble choosing our project idea because we saw many risks associated with our previous ideas. With this project, the testing and the use of libraries should not be difficult as we already have the code. However, our main risk is that two of us have no experience using Android Studio while one of us knows a little bit about it. \sout{All in all, we are very new to using Android Studio which may cause a few issues. However, we plan on doing a lot of research online to play around with Android Studio. If we see that after one month, we still feel uncomfortable with Android Studio, we will reduce our scope and get more help from the TAs.} {\color{red} During the Proof Of Concept demonstration with the TA and Professor, we are planning to show a working application on Android which allows the user to take a picture. If we can get this done, it would display that we have become more familiarized with Android development and Android Studio. Along with this demo, we will use our Gantt chart to create checkpoints of what implementation we should have done by which date.}

\section{Technology}
We will be coding with Java using Android Studio to create our Android application.\sout{, while using Eclispe as our IDE.} To generate our documents, we will be using Latex. We will be using JUnit to test our code and to test the actual application, we will be using an Android device. We will also be using integration testing with Android Studio.

\section{Coding Style}
We will adhere to the standard Java coding style for our project, which will follow the correct conventions for naming, file structure, formatting, and correct programming practices. {\color{red} {The link for our coding style can be found \href{http://www.oracle.com/technetwork/java/codeconvtoc-136057.html}{here:}}}

\section{Project Schedule}

Please see our Gantt chart \href{https://gitlab.cas.mcmaster.ca/kowserm/3xa3/blob/master/Doc/Design/Gantt%20Chart/project_gantt.gan}{here}

\section{Project Review}
{\color{red} Doing this project was a great learning experience as we all pushed ourselves to learn more than we had actually planned. We initially started with a few features but once we got into the implementation, we kept trying to add more features which led us to learn more about Android mobile developement and Android Studio. Even the documents, although very boring, were a good learning experience as it allowed us to see how an actual real-life project would be created from scratch. As a group, we think we did well and achieved mainly what we wanted to out this project, but we also could have done much better. We communicated well throughout the term and took up appropriate roles from the beginning of the project so we could work with our strengths. We believe that more regular meetings would have allowed for better documents which are not submitted as rushed products. In the future, we the only the we would modify would be to be more organized so that we have better documents, a more smooth presentation, and an even better learning experience. This could be achieved through better planning in the beginning stages of the project where could decide definitive meeting dates and plans for those days starting from that time to the end of the course. We did try to meet up regularly, but not being able to foresee what other work and commitments we had really pushed us back.}

\end{document}