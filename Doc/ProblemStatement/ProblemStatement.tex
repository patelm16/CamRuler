\documentclass{article}

\usepackage{tabularx}
\usepackage{booktabs}

\title{SE 3XA3: Problem Statement\\CamRuler}

\author{Team 10
		\\ Kshitij Mehta, mehtak1
		\\ Prince Kowser, kowserm
		\\ Meet Patel, patelm16
}

\date{\today}

\begin{document}

\begin{table}[hp]
\caption{Revision History} \label{TblRevisionHistory}
\begin{tabularx}{\textwidth}{llX}
\toprule
\textbf{Date} & \textbf{Developer(s)} & \textbf{Change}\\
\midrule
September 25, 2017 & Kshitij Mehta, Meet Patel, Prince Kowser & Rev 0\\
\bottomrule
\end{tabularx}
\end{table}

\newpage

\maketitle

\section{What problem are you trying to solve? }
Often times, we try to do a task and cannot do it simply because we do not have the resources to do so. One such problem is measuring an object. The app we are re-implementing is a camera ruler, which allows the user to measure an object in front of them through a picture of the object and a reference object. Having this app on mobile devices will enable users to conveniently measure objects without the struggle of finding a ruler and doing tedious measurements by hand. 

\section{Why is this an important problem?}
This is an important problem because often times in class, labs, or work that requires measuring, we need to find some precise measurements. This app allows easier access to a precise measuring resource. People may forget to bring their measuring devices or do not want to carry around the devices. Our app allows the user to measure objects with just their phone which almost everyone carries around. All that is required is another object that they know the measurement off which can be a paper, pen, etc. A big advantage of our app is that for a ruler or a measuring tape has a limit to their measurement however in our app, if you can fit the object in your picture then you can measure it. All these benefits make this app very convenient for work that require a lot of measurements.

\section{What is the context of the problem you are solving?}
Our re-implementation of Camera Ruler will be very beneficial to users as it will have better accuracy, easier usability and multiple objects measurement.
\subsection{Who are the stakeholders?}
The stakeholders of this app would be the original software developers, designers, architects, students, and anyone that needs to conduct measurements. People who are working in labs and need to measure the length of multiple objects with almost no uncertainty for their calculations would be great end users. Also carpenters or someone who uses a woodshop would also find our app very usefull. 
\subsection{What is the environment for the software?}
This is a mobile application which will be built on Windows using Java and we will be targeting mobile users with an Android operating system. It will be accessible by any user who has an android phone removing the hassle of carrying any measuring devices and does not require help measuring something very lengthy. If it can fit in a picture then it is measurable.
\end{document}